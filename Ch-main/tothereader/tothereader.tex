\documentclass[main.tex]{subfiles}



\setlength{\columnsep}{30pt}

\begin{document}
\pagenumbering{Roman}
\pagestyle{style1}
\newgeometry{left=0.8in,right=2.8in, top=3.5cm,bottom=2cm}

\begin{fullwidth}
\begin{multicols*}{2}
The work of chemists is certainly challenging, but also exciting and rewarding. Chemists produce everything from plastics and paints to pharmaceuticals, foods, flavors, fragrances, detergents, and cosmetics. Chemistry students are well-prepared for medical, veterinarian, dentistry, optometry, or pharmacy school. This set of lecture notes was designed with a focus on the student--with a focus on you. It introduces the basic concepts of college chemistry in a way that a student of any level can hopefully understand. Some of the chapters included in this guide can be challenging. Success is not an accident. Only with hard work, patience, and perseverance, you will be able to achieve what you want. 

Chemistry is not an easy subject. You may experience frustration due to the terminology or the math content. This guide is developed in chapters and sections to break down the very basics of chemistry concepts. One of the main goals here is to help you solve chemistry problems. Solving problems--not only chemistry problems but problems of any kind--is an extremely useful skill in life. Chemists approach the solving of problems in a very specific way. They use critical thinking and previous knowledge to find solutions based on the information presented. As you study this set of lectures, I encourage you to read the different sections of a chapter, highlight the main ideas, and find keywords that represent new concepts. Numerous examples are presented in the chapters with the full solution. A lot of examples are also presented without the worked solution, just including the answer. Plenty of end of the chapter problems is further included. After you read the content of a chapter you are highly encouraged to work on the end of the chapter problems. As with any skill, practice makes perfect.

There are numerous tools in this guide to help you focus on the most relevant content. For example, when the numerical problems get too complex, an \emph{analyze the problem box} is included to help you identify what is given and what is asked in the problem.

This set of lectures resonates with the open textbook movement that is taking over CUNY as well as SUNY. Education is expensive and you as a student often rely on textbooks to learn. These valuable educational resources are often used for a very limited period and tossed or returned when a class has finished. The open textbook movement aims to alleviate the cost of education by relying on resources that are free for both the students and the educators. Still, these sources are imperfect and not as curated as textbooks, and this is the price to pay. I warn you this set of lectures is indeed imperfect, hence its title. Yet, it is the result of many hours of work--indeed months of work. Your role is key. I encourage you first to be understanding and patient, and then to contribute to the development of this guide. With your input, we can make this guide a better educational resource. Mind that this guide was written by an educator and as such, it sometimes uses terms and a way of thinking that corresponds to the educators\textquotesingle point of view.

This set of lectures does not intend to replace any textbook. Indeed, there are many high-quality textbooks in the literature that I recommend. For College Chemistry:
\begin{small}\begin{itemize}[label=\resizebox{!}{.7em}{\rotatebox[origin=c]{-90}{\includegraphics{./tothereader/lis}}}]
\setlength\itemsep{0.5em}
\item Chemical Principles: The Quest for Insight by Peter Atkins et al.
\item Chemistry: The Central Science by Theodore E. Brown et al.
\item Chemistry by Steven S. Zumdahl et al.
\item Chemistry: The Molecular Nature of Matter and Change by Martin Silberberg et al.
\item Chemistry by Raymond Chang et al.
\item Chemistry: Atoms First by OpenStax
\end{itemize}\end{small}

For GOB Chemistry:
\begin{small}\begin{itemize}[label=\resizebox{!}{.7em}{\rotatebox[origin=c]{-90}{\includegraphics{./tothereader/lis}}}]
\setlength\itemsep{0.5em}\item General, Organic, and Biological Chemistry: Structures of Life by Karen C. Timberlake et al.
\end{itemize}\end{small}

With the help of the textbooks above you can certainly expand and complement the information presented in this guide.

This guide was fully coded in \LaTeX from the cover or the periodic table to the molecular orbital diagrams or the solid representations. Chemistry is a microscopic science not accessible to the naked eye. Visuals play a very important role in chemistry education. Visuals--in the form of images or diagrams--help make chemistry more apparent to the viewer. One of the weak points of many open education chemistry guides is the visuals. They tend to be simplistic with low quality. This guide extensively relies on images and diagrams and uses the Tikz software package and other open-source tools to freshly generate diagrams every time the book is compiled. All other images used here are open-source images.

We hope you enjoy this guide and more importantly, that it contributes to your career success.  
%\par \medskip
%\includegraphics[height=4.5\baselineskip,]{./tothereader/signature} \par
%Daniel Torres \par
%Brooklyn
\end{multicols*}
\clearpage\thispagestyle{empty}\mbox{}\clearpage

\end{fullwidth}
\restoregeometry
\pagenumbering{arabic}
\end{document}
