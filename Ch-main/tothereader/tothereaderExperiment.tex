\documentclass[main.tex]{subfiles}



\setlength{\columnsep}{30pt}

\begin{document}
\pagenumbering{Roman}
\pagestyle{style1}
\newgeometry{left=0.8in,right=2.8in, top=3.5cm,bottom=2cm}

\begin{fullwidth}
\begin{multicols*}{2}


Dear Colleagues\\
	It is with great pleasure that we offer the alpha edition of \begin{it}Experiments in College Chemistry I: an Opentextbook\end{it}. This is, to the best of the author's knowledge, the first opentextbook of experimental chemistry at a entry-college level.\\
	Chemistry embraces everything material around us such as the clothes we wear, the air we breath or the batteries that power our cellphone, and today's chemistry is built on centuries of exploration and discovery. Most of the College Chemistry textbooks in the market  contain useful information about the basic principles of chemistry thats is needed for the majority of the science and engineering student, even after graduation. Whether it's a glossary of chemical terms, or basic information about the chemistry principles, it is extremely handy to have those books around for further reference. There are also many good college chemistry laboratory manuals in the market result of the work of devoted faculty. They normally contain a series of laboratory procedures or inquiry-based experiments developed and carefully tested throughout the years. Nevertheless, these are rarely reference material as they have instead a more limited use. Students extensively use these manuals, sometimes writing their measurements in worksheets attached, what makes impossible to reuse them. But more importantly: laboratory manuals are expensive. With an average price of \$50, they represent a considerable investment for a student just for single-use purchase. With all these in mind \begin{it}Experiments in College Chemistry I: an Opentextbook\end{it} aims to alleviate these burdens by providing a set of standard chemistry experiments freely available for faculty and students.\\
	The main part of the experiments in this book deal with classical chemistry techniques such as reading a meniscus, making dilutions or performing a filtration or a titration. These are standard techniques and experiments such as the acetic acid titration have been extensively validated in the literature. The reactants employed in the experiment are standard chemicals easily accessible in any general chemistry laboratory and the safety measures are minimal.\\
	I like to think that this this opentexbook is designed to encourage students to think and to develop a solid understanding of chemistry by first building a qualitative understanding and then practicing basic chemistry techniques. Because college students often have forgotten much of their high school chemistry, each experiment begins with a background section that reviews the basic ideas behind the experiment, providing sometimes worked examples to illustrate the concepts. \\
We are aware of the difficulty that students have with math, as well as with plotting and graphing. With that in mind, we have included several plotting exercises in the labs that will help students practice these techniques. \\
%We are also fully aware that students find quantum theory and atomic structure daunting and there are limited experiments available in the literature to cover those subjects. To make this material more accessible, we have developed a series of modeling experiments covering electronic structure, molecular structure, chemical bond and the study of solids. These theory experiments, introducing the use of chemistry modeling to the lower levels chemistry students, use the freeware GPAW and ASE.
Last, but by no means least, the author acknowledges the presence of many typos and mistakes and that is the reason this represents the \begin{it}alpha \end{it} edition of the manual. Any critical input from the reader will be well-received.\\
The authors.\\
NYC, January 2017

\end{multicols*}
\clearpage\thispagestyle{empty}\mbox{}\clearpage

\end{fullwidth}
\restoregeometry
\pagenumbering{arabic}
\end{document}
