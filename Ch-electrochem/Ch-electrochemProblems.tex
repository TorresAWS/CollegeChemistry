\documentclass[main.tex]{subfiles}
\newcommand\chapterlabel{Ch-electrochem}
\begin{document}\newpage
 
\newgeometry{left=0.8in,right=0.8in, top=2.5cm,bottom=2cm}
\fancyhfoffset[E,O]{0pt}
\setlength{\columnsep}{30pt}
\begin{conclusion}
\end{conclusion}
%\setstretch{0.3}
\begin{multicols*}{2}\setcounter{numA}{1}
{\raggedright\textsc{\textbf{Introduction to galvanic cells}}\par} \import{../\chapterlabel/problems/problems/}{Problems-Introduction-to-galvanic-cells}
{\raggedright\textsc{\textbf{Standard reduction potentials}}\par} \import{../\chapterlabel/problems/problems/}{Problems-Standard-reduction-potentials}
{\raggedright\textsc{\textbf{Redox reactions in galvanic cells}}\par} \import{../\chapterlabel/problems/problems/}{Problems-Redox-reactions-in-galvanic-cells}
{\raggedright\textsc{\textbf{Line notation for galvanic cells}}\par} \import{../\chapterlabel/problems/problems/}{Problems-Line-notation-for-galvanic-cells}
{\raggedright\textsc{\textbf{Cell potential, Gibbs free energy, and equilibrium constant}}\par} \import{../\chapterlabel/problems/problems/}{Problems-Cell-potential-Gibbs-free-energy-and-equilibrium-constant}
{\raggedright\textsc{\textbf{Electrochemical series: dissolving metals in acid}}\par} \import{../\chapterlabel/problems/problems/}{Problems-Electrochemical-series}
{\raggedright\textsc{\textbf{Nernst equation}}\par} \import{../\chapterlabel/problems/problems/}{Problems-Nernst-equation}
{\raggedright\textsc{\textbf{Electrolysis}}\par} \import{../\chapterlabel/problems/problems/}{Problems-Electrolysis}



\end{multicols*}
\newpage
\begin{answersenvironment}
\begin{minipage}[c]{1\textwidth}
\begin{localsize}{10}
{\Large \bf Answers}
\SetupExSheets{
  headings = inline-nr , % numbered and inline
  counter-format = qu) , % numbers 1) 2) ... 
}
 \printsolutions 
 % \printsolutions[byID={1,3,5,7,9,11,13,15,17,19,21,23,25,27,29,31,33,35,37, 39, 41, 43, 45, 47, 49, 51,53,55,57,59}]
\end{localsize}
\end{minipage}\end{answersenvironment}
\end{document}
