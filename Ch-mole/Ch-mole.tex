\documentclass[main.tex]{subfiles}
\newcommand\chapterlabel{Ch-mole}\setcounter{figurenewcounter}{0}\setcounter{tablenewcounter}{0}\setcounter{formulanewcounter}{0}

\begin{document}

\linenumbers


\chapter[The Mole and Chemical Reactions ]{The Mole and Chemical Reactions}

    \begin{marginfigure}
\begin{tikzpicture} \node (a) at (0,0) {\includegraphics[width=4cm]{../{\chapterlabel}/figure1}} node[rotate=90, font=\tiny] at ([yshift=.5cm,xshift=.1cm]a.south east) {\textsuperscript{\textcopyright} PngImg} ;
\end{tikzpicture}
\end{marginfigure}



\lettrine[lines=4]{\color{black!45}W}{hen} we buy eggs in the store, we buy them by the dozen, and the word dozen actually refers to the number twelve. Similarly, when we measure substances in a chemistry lab we measure them by the mole. This chapter will introduce the idea of mole and you will learn how to relate moles of a chemical to mass using a property called the molecular mass. This chapter also introduces chemical reactions. Chemicals react with each others and a chemical reaction is written in the form an equations. In this chapter you will learn how to balance those equations in order to predict the amount of chemicals produced.

\begin{marginfigure}%LEARNING GOALS BOX
\begin{mytcbox}{GOALS}
\begin{enumerate}[label=\protect\circled{\color{white}\arabic*}]
\item Transform grams into moles and moles into molecules
\item Balance chemical reactions
\item Carry stoichiometric calculations
\item Identify the limiting reagent
\item Calculate the \% yield
\end{enumerate}
\end{mytcbox}
\vspace{1cm}
\begin{tcolorbox}[enhanced,colback=red!5!white,colframe=black!50!red,boxrule=1pt,
  arc=0pt,outer arc=0pt,drop heavy lifted shadow]
\faGears\ 
\docenvdef{Discussion:} What weights more one kilo of Sulfur or one kilo of Gold? Now, what weights more, one mol of Sulfur or one mole of Gold
 \end{tcolorbox}
\end{marginfigure}%LEARNING GOALS BOX





\section{The mole}
\import{files/}{SectionIntro-The-mole}

\sloppy \begin{description}

\item[\docfilehook{From moles to molecules}{}] \import{files/}{SubSection-The-mole-From-moles-to-atoms}
\item[\docfilehook{From molecules to atoms}{}] \import{files/}{SubSection-The-mole-From-molecules-to-atoms}
  \hspace{-3cm}\import{files/}{Figure-The-mole}
\import{problems/}{SampleProblem1}
\import{problems/}{SampleProblem10}

\import{files/}{SideFigure-Molar-mass}

\end{description}
	





\section{Converting moles into grams and into atoms}
\import{files/}{SectionIntro-Converting-moles-into-grams-and-into-atoms}


\sloppy \begin{description}

\item[\docfilehook{Molar mass of a chemical}{}] \import{files/}{SubSection-Converting-moles-into-grams-and-into-atoms-Molar-mass-of-a-chemical}




\import{problems/}{SampleProblem2}
\import{files/}{SubSection-Converting-moles-into-grams-and-into-atoms-From-moles-to-grams}








 \hspace{2cm}\import{files/}{Figure-Mole-equivalency}





\import{problems/}{SampleProblem3}

 

\import{files/}{SubSection-Converting-moles-into-grams-and-into-atoms-From-grams-to-atoms}


\import{files/}{SideFigure-Reactions}
\import{problems/}{SampleProblem4}


\end{description}






\section{Chemical reactions}
\import{files/}{SectionIntro-Chemical-reaction}

\sloppy\begin{description}


\import{files/}{SubSection-Chemical-reaction-Simple-chemical-reactions}
\import{files/}{SubSection-Chemical-reaction-Reading-a-chemical-reaction}




\import{files/}{SubSection-Chemical-reaction-Balanced-chemical-reactions}
\import{files/}{SubSection-Chemical-reaction-Balancing-chemical-reactions}





\import{problems/}{SampleProblem5}


\import{files/}{SubSection-Chemical-reaction-Five-types-of-reactions}


\end{description}



\section{Stoichiometry and mass calculations}
\import{files/}{SectionIntro-Stoichiometry-and-mass-calculations}


\sloppy \begin{description}

\import{files/}{SubSection-Stoichiometry-and-mass-calculations-Mole-Mole-ratio}
\import{files/}{SubSection-Stoichiometry-and-mass-calculations-Reactants-to-products}
\import{files/}{SubSection-Stoichiometry-and-mass-calculations-Reactant-to-a-different-reactant}








\import{problems/}{SampleProblem6}
\import{files/}{SubSection-Stoichiometry-and-mass-calculations-Mass-calculations}

\import{files/}{SubSection-Stoichiometry-and-mass-calculations-Molar-mass-review}


\import{files/}{SubSection-Stoichiometry-and-mass-calculations-Grams-to-moles-in-a-reaction}




\import{files/}{SubSection-Stoichiometry-and-mass-calculations-Grams-to-grams-in-a-reaction}


\import{problems/}{SampleProblem7}

\end{description}

 








\section{Percent yield and limiting reagent}
\import{files/}{SectionIntro-Percent-yield-and-limiting-reagent}

\sloppy \begin{description}

\import{files/}{SubSection-Percent-yield-and-limiting-reagent-Percent-yield}



\import{problems/}{SampleProblem8}
\import{files/}{SubSection-Percent-yield-and-limiting-reagent-Limiting-reagent}
\import{files/}{SubSection-Percent-yield-and-limiting-reagent-Identifying-the-limiting-reagent}



 


\import{problems/}{SampleProblem9}

\end{description}




\end{document}

