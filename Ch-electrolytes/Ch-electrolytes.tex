\documentclass[main.tex]{subfiles} %NAME=electrolytes
\newcommand\chapterlabel{Ch-electrolytes}\setcounter{figurenewcounter}{0}\setcounter{tablenewcounter}{0}\setcounter{formulanewcounter}{0}

\begin{document}

\linenumbers


\import{files/}{ChapterName}


 \begin{marginfigure}
\begin{tikzpicture} \node (a) at (0,0) {\includegraphics[width=4cm]{../{\chapterlabel}/figure1}} node[rotate=90, font=\tiny] at ([yshift=.5cm,xshift=.1cm]a.south east) {\textsuperscript{\textcopyright} PngImg} ;
\end{tikzpicture}
\label{fig:electrolytes1}
\end{marginfigure}
\import{files/}{ChapterIntro}
\begin{marginfigure}%LEARNING GOALS BOX
\begin{mytcbox}{GOALS}
\begin{enumerate}[label=\protect\circled{\color{white}\arabic*}]
\import{files/}{SectionGoal-An-introduction-to-reactions-in-solution}
\import{files/}{SectionGoal-Solutions-and-composition}
\import{files/}{SectionGoal-Electrolytes-and-insoluble-compounds}
\import{files/}{SectionGoal-Precipitation-reactions-and-acid-base-reactions}
\import{files/}{SectionGoal-Redox-reactions}
\end{enumerate}
\end{mytcbox}
\vspace{1cm}
\begin{tcolorbox}[enhanced,colback=red!5!white,colframe=black!50!red,boxrule=1pt,
  arc=0pt,outer arc=0pt,drop heavy lifted shadow]
\faGears\ 
\import{files/}{ChapterDiscussion}
\end{tcolorbox}
\end{marginfigure}%LEARNING GOALS BOX




\section{\color{blue!30!black}{Solutions and composition}}
\import{files/}{SectionIntro-Solutions-and-composition}
\sloppy\begin{description}
\item[\docfilehook{What makes a solution?}{}]\import{files/}{SubSection-Solutions-and-composition-What-makes-a-solution}
\import{problems/}{SampleProblem1}
\item[\docfilehook{Types of solutions}{}] \import{files/}{SubSection-Solutions-and-composition-Types-of-solutions}
 \hspace{-0cm}\import{files/}{Figure-Dissolution}
\item[\docfilehook{Empirical rules of polarity}{}] \import{files/}{SubSection-Solutions-and-composition-Empirical-rules-of-polarity}
\import{problems/}{SampleProblem2}
\item[\docfilehook{Mixing and polarity}{}]\import{files/}{SubSection-Solutions-and-composition-Mixing-and-polarity}
 \hspace{2cm}\import{files/}{Table-Polarity-and-mixing}	
\import{problems/}{SampleProblem3}
\item[\docfilehook{Concentration of solutions}{}]
%\import{files/}{SubSection-Solutions-and-composition-Concentration-of-solutions}
\import{files/}{SubSection-Solutions-and-composition-Meaning-of-concentration}
\item[\docfilehook{Mass percent concentration}{}]\import{files/}{SubSection-Solutions-and-composition-Mass-percent}
\import{problems/}{SampleProblem4}
\item[\docfilehook{Volume percent concentration}{}] \import{files/}{SubSection-Solutions-and-composition-Volume-percent-concentration}
\item[\docfilehook{Mass/volume  percent concentration}{}]\import{files/}{SubSection-Solutions-and-composition-Massvolume-percent-concentration}
\item[\docfilehook{Molarity concentration}{}]\import{files/}{SubSection-Solutions-and-composition-Molarity-concentration}
\import{problems/}{SampleProblem5}


\item[\docfilehook{Concentration units as conversion factors}{}]\import{files/}{SubSection-Solutions-and-composition-Concentration-units-as-conversion-factors}
\import{problems/}{SampleProblem6}
\item[\docfilehook{Dilution}{}]\import{files/}{SubSection-Solutions-and-composition-Dilution}
\import{problems/}{SampleProblem7}
\item[\docfilehook{Molarity calculation involving grams}{}]\import{files/}{SubSection-Solutions-and-composition-Molarity-calculations-with-mass}

\end{description}










\section{\color{blue!30!black}{Electrolytes and insoluble compounds}}
\import{files/}{SectionIntro-Electrolytes-and-insoluble-compounds}
\sloppy \begin{description}
\item[\docfilehook{Soluble and insoluble salts}{}] \import{files/}{SubSection-Electrolytes-and-insoluble-compounds-Soluble-and-insoluble-salts}
\import{problems/}{SampleProblem8}
\vspace{0cm}\import{files/}{Table-Solubility}
\item[\docfilehook{Strong electrolytes}{}]\import{files/}{SubSection-Electrolytes-and-insoluble-compounds-Strong-electrolytes}
\item[\docfilehook{Nonelectrolytes}{}]\import{files/}{SubSection-Electrolytes-and-insoluble-compounds-Nonelectrolytes}

\item[\docfilehook{Identify the electrolyte character of a chemical}{}]\import{files/}{SubSection-Electrolytes-and-insoluble-compounds-Identify-the-electrolyte-character-of-a-chemical}
 \vspace{1cm}\import{files/}{Table-Electrolytes	}	

\item[\docfilehook{Breaking down electrolytes into ions}{}]\import{files/}{SubSection-Electrolytes-and-insoluble-compounds-Breaking-down-electrolytes-into-ions}
\import{problems/}{SampleProblem9}
\import{problems/}{SampleProblem10}
\end{description}


\section{\color{blue!30!black}{An introduction to reactions in solution}}
\import{files/}{SectionIntro-An-introduction-to-reactions-in-solution}
\sloppy \begin{description}
\item[\docfilehook{Acid-base reactions}{}] \import{files/}{SubSection-An-introduction-to-reactions-in-solution-Acid-base-reactions}
\item[\docfilehook{Precipitation reactions}{}]\import{files/}{SubSection-An-introduction-to-reactions-in-solution-Precipitation-reactions}
\item[\docfilehook{Redox reactions}{}]\import{files/}{SubSection-An-introduction-to-reactions-in-solution-Redox-reactions}
\import{problems/}{SampleProblem11}
\end{description}

\section{\color{blue!30!black}{Precipitation reactions and acid-base reactions}}
\import{files/}{SectionIntro-Precipitation-reactions-and-acid-base-reactions}
\sloppy \begin{description}
\item[\docfilehook{Solubility formula}{}] \import{files/}{SubSection-Precipitation-reactions-and-acid-base-reactions-Solubility-formula}
\item[\docfilehook{Acid-base reactions}{}]\import{files/}{SubSection-Precipitation-reactions-and-acid-base-reactions-Acid-base-reactions}
\item[\docfilehook{Precipitation reactions}{}] \import{files/}{SubSection-Precipitation-reactions-and-acid-base-reactions-Precipitation-reactions}
\item[\docfilehook{Formula equations, ionic equations and net ionic equations}{}]\import{files/}{SubSection-Precipitation-reactions-and-acid-base-reactions-Formula-equations-ionic-equations-and-net-ionic-equations}
\import{problems/}{SampleProblem12}
\end{description}

\section{\color{blue!30!black}{Redox reactions}}
\import{files/}{SectionIntro-Redox-reactions}
\sloppy \begin{description}
\item[\docfilehook{Oxidation state or redox number}{}]\import{files/}{SubSection-Redox-reactions-Oxidation-state-or-redox-number}
 \item[\docfilehook{Rules to calculate redox numbers}{}]\import{files/}{SubSection-Redox-reactions-Rules-to-calculate-redox-numbers}
 \item[\docfilehook{Calculating the redox number}{}]\import{files/}{SubSection-Redox-reactions-Calculating-the-redox-number}
 \import{problems/}{SampleProblem13}
\item[\docfilehook{Redox means oxidation and reduction}{}]\import{files/}{SubSection-Redox-reactions-Redox-means-oxidation-and-reduction}
 \item[\docfilehook{Redox numbers in chemical reactions}{}]\import{files/}{SubSection-Redox-reactions-Redox-numbers-in-chemical-reactions}
 \item[\docfilehook{Balancing simple redox reactions}{}] \import{files/}{SubSection-Redox-reactions-Balancing-simple-redox-reactions}
\item[\docfilehook{Balancing redox reactions in acidic medium}{}] \import{files/}{SubSection-Redox-reactions-Balancing-redox-reactions-in-acidic-medium}
\import{problems/}{SampleProblem14}
\item[\docfilehook{Balancing redox reactions in basic medium}{}] \import{files/}{SubSection-Redox-reactions-Balancing-redox-reactions-in-basic-medium}
\end{description}




\end{document}
