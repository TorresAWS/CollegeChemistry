\documentclass[main.tex]{subfiles}
\begin{document}\newpage
\setdoublesep{0.35700 em}  % 'Bond Spacing'
\setatomsep{1.78500 em}    % 'Fixed Length'
\setbondoffset{0.18265 em} % 'Margin Width'
\newcommand{\bondwidth}{0.06642 em} % 'Line Width'
\setbondstyle{line width = \bondwidth}
\newgeometry{left=0.8in,right=0.8in, top=2.5cm,bottom=2cm}
\fancyhfoffset[E,O]{0pt}
\setlength{\columnsep}{30pt}
\begin{conclusion}
\end{conclusion}
\setstretch{0.3}
\begin{multicols*}{2}


{\raggedright\textsc{\textbf{Radiation \& particles }}\par}


\begin{enumerate}

\item Identify the nuclear symbol for iodine-134:
\begin{enumerate}[label=(\alph*)]
\begin{multicols*}{2}
\item $\alpha $
\item \ce{^133_53Y}
\item \ce{^134_53I}
\item \ce{^134_56I}
\item \ce{^134_53Io}
\end{multicols*}\flushright  {\small Ans: (c)}
\end{enumerate}

\item Calculate the number of electrons of \ce{^24_17Mg} :
\begin{enumerate}[label=(\alph*)]
\begin{multicols*}{2}
\item $\alpha $
\item 24
\item 12
\item 17
\item 7
\end{multicols*}\flushright  {\small Ans: (d)}
\end{enumerate}

\item Calculate the number of neutrons of \ce{^24_17Mg} :
\begin{enumerate}[label=(\alph*)]
\begin{multicols*}{2}
\item 3
\item 24
\item 12
\item 17
\item 7
\end{multicols*}\flushright  {\small Ans: (e)}
\end{enumerate}

\item Calculate the number of protons of \ce{^24_17Mg} :
\begin{enumerate}[label=(\alph*)]
\begin{multicols*}{2}
\item 2
\item 24
\item 12
\item 17
\item 7
\end{multicols*}\flushright  {\small Ans: (d)}
\end{enumerate}


\item The nuclear symbol for \ce{^4_2He} also refers to:
\begin{enumerate}[label=(\alph*)]
\begin{multicols*}{2}
\item $\alpha$ particle
\item $\beta$ particle
\item $\gamma$ particle
\item positron
\item proton
\end{multicols*}\flushright  {\small Ans: (a)}
\end{enumerate}
\item The nuclear symbol for  \ce{^0_-1e} also refers to:
\begin{enumerate}[label=(\alph*)]
\begin{multicols*}{2}
\item $\alpha$ particle
\item $\beta$ particle
\item $\gamma$ particle
\item positron
\item proton
\end{multicols*}\flushright  {\small Ans: (b)}
\end{enumerate}
\item The nuclear symbol for $\beta^+$ also refers to:
\begin{enumerate}[label=(\alph*)]
\begin{multicols*}{2}
\item $\alpha$ particle
\item $\beta$ particle
\item $\gamma$ particle
\item positron
\item proton
\end{multicols*}\flushright  {\small Ans: (d)}
\end{enumerate}


{\raggedright\textsc{\textbf{Nuclear Reaction }}\par}

\item The nuclear reaction below is an example of what type of nuclear reaction? \\
\begin{center}
\ce{^238_92U -> ^234_90Th + ^4_2He} 
\end{center}
\begin{enumerate}[label=(\alph*)]
\begin{multicols*}{2}
\item $\alpha$ decay
\item $\beta$ decay
\item $\gamma$ decay
\item positron emission
\end{multicols*}\flushright  {\small Ans: (a)}
\end{enumerate}

\item The nuclear reaction below is an example of what type of nuclear reaction? \\
\begin{center}\ce{^42_19K -> ^42_20Ca + ^0_{-1}e} \end{center}
\begin{enumerate}[label=(\alph*)]
\begin{multicols*}{2}
\item $\alpha$ decay
\item $\beta$ decay
\item $\gamma$ decay
\item positron emission
\end{multicols*}\flushright  {\small Ans: (b)}
\end{enumerate}


\item The nuclear reaction below is an example of what type of nuclear reaction? \\
\begin{center}\ce{^15_8O -> ^15_7N + ^0_{+1}e} \end{center}
\begin{enumerate}[label=(\alph*)]
\begin{multicols*}{2}
\item $\alpha$ decay
\item $\beta$ decay
\item $\gamma$ decay
\item positron emission
\end{multicols*}\flushright  {\small Ans: (d)}
\end{enumerate}

\item The nuclear reaction below is an example of what type of nuclear reaction? \\
\begin{center}\ce{^228_88Ra -> ^228_89Ac + ^0_{-1}e} \end{center}
\begin{enumerate}[label=(\alph*)]
\begin{multicols*}{2}
\item $\alpha$ decay
\item $\beta$ decay
\item $\gamma$ decay
\item positron emission
\end{multicols*}\flushright  {\small Ans: (b)}
\end{enumerate}

\item The nuclear reaction below is an example of what type of nuclear reaction? \\
\begin{center}\ce{^13_6C + ^1_1H -> ^14_7N + ^0_0\gamma}\end{center} 
\begin{enumerate}[label=(\alph*)]
\begin{multicols*}{2}
\item $\alpha$ decay
\item $\beta$ decay
\item $\gamma$ decay
\item positron emission
\end{multicols*}\flushright  {\small Ans: (c)}
\end{enumerate}

{\raggedright\textsc{\textbf{Unknown isotopes }}\par}

\item What is the radioactive particle involved in the following nuclear equation? \\
\begin{center}\ce{^9_4Be + ^A_ZX -> ^12_6C + ^1_0n} \end{center} 
\begin{enumerate}[label=(\alph*)]
\begin{multicols*}{2}
\item $\alpha$ particle
\item $\beta$ particle
\item $\gamma$ particle
\item positron 
\item neutron 
\end{multicols*}\flushright  {\small Ans: (a)}
\end{enumerate}






\item What is the radioactive particle involved in the following nuclear equation? \\
\begin{center}\ce{^31_15P + ^1_1H -> ^31_16S + ^A_ZX} \end{center} 
\begin{enumerate}[label=(\alph*)]
\begin{multicols*}{2}
\item $\alpha$ particle
\item $\beta$ particle
\item $\gamma$ particle
\item positron 
\item neutron 
\end{multicols*}\flushright  {\small Ans: (e)}
\end{enumerate}

\item What is the radioactive particle involved in the following nuclear equation? \\
\begin{center}\ce{^3_1H + ^2_1H  -> ^A_ZX + ^1_0n}  \end{center} 
\begin{enumerate}[label=(\alph*)]
\begin{multicols*}{2}
\item $\alpha$ particle
\item $\beta$ particle
\item $\gamma$ particle
\item positron 
\item neutron 
\end{multicols*}\flushright  {\small Ans: (a)}
\end{enumerate}

{\raggedright\textsc{\textbf{Half-Life }}\par}



\item  Xenon-133, which is used for lung imaging, has a half-life of 5.2 days.  If 50.0 mg of Xe-133 were prepared at 8:00 A.M. on Monday, how many mg remain at 8:00 A.M. on the following day?
\begin{enumerate}[label=(\alph*)]
\begin{multicols*}{2}
\item 10 mg
\item 44 mg
\item 50 mg
\item 40 mg
\item 35 mg
\end{multicols*}\flushright  {\small Ans: (b)}
\end{enumerate}
\item  Gold-198, which is used for liver disease diagnosis, has a half-life of 2.7 days.  If 100.0 mg of Au-198 were prepared at 8:00 A.M. on Monday, how many mg remain at 8:00 A.M. on Wednesday?
\begin{enumerate}[label=(\alph*)]
\begin{multicols*}{2}
\item 1 mg
\item 10 mg
\item 100 mg
\item 50 mg
\item 40 mg
\end{multicols*}\flushright  {\small Ans: (e)}
\end{enumerate}

\item  Gold-198, which is used for liver disease diagnosis, has a half-life of 2.7 days.  If 100.0 mg of Au-198 were prepared at 8:00 A.M. on Monday, how many mg remain at 2:00 P.M. on Wednesday?
\begin{enumerate}[label=(\alph*)]
\begin{multicols*}{2}
\item 10 mg
\item 56 mg
\item 45 mg
\item 40 mg
\item 53 mg
\end{multicols*}\flushright  {\small Ans: (b)}
\end{enumerate}

\item  The half-life of bromine-74 is 25 min. How much of a 100 mg sample is still active after 100 min?
\begin{enumerate}[label=(\alph*)]
\begin{multicols*}{2}
\item 10 mg
\item 6 mg
\item 8 mg
\item 10 mg
\item 2 mg
\end{multicols*}\flushright  {\small Ans: (b)}
\end{enumerate}



\item  The half-life of bromine-74 is 25 min. 20mg of the isotopes remain after 10 minutes of preparing the sample. Calculate the initial mass of the bromine-74 sample.
\begin{enumerate}[label=(\alph*)]
\begin{multicols*}{2}
\item 30 mg
\item 26 mg
\item 20 mg
\item 40 mg
\item 10 mg
\end{multicols*}\flushright  {\small Ans: (b)}
\end{enumerate}

\item  The half-life of Au-198 is 2.7 days. 100mg of the isotopes remain after 5days of preparing the sample. Calculate the initial mass of the isotope sample.
\begin{enumerate}[label=(\alph*)]
\begin{multicols*}{2}
\item 400 mg
\item 361 mg
\item 300 mg
\item 234 mg
\item 100 mg
\end{multicols*}\flushright  {\small Ans: (b)}
\end{enumerate}

%\item  The half-life of Au-198 is 2.7 days. We prepare a 100mg sample and after a certain time, the mass of the sample became 50mg. How much time has passed since the sample preparation?
%\begin{enumerate}[label=(\alph*)]
%\begin{multicols*}{2}
%\item 2.7days
%\item 5days
%\item 1day
%\item 100 days
%\item 3 days
%\end{multicols*}\flushright  {\small Ans: (a)}
%\end{enumerate}



{\raggedright\textsc{\textbf{Radiation measurement}}\par}


\item  \ce{^199Tc^*} is a radioisotope used for liver disease diagnosis. The administered  activity of the isotope is 740MBe. How much is this activity in mCi?
\begin{enumerate}[label=(\alph*)]
\begin{multicols*}{2}
\item 10mCi
\item 20mCi
\item 30mCi
\item 40mCi
\item 50mCi
\end{multicols*}\flushright  {\small Ans: (b)}
\end{enumerate}


\item  \ce{^201Tl^*} is a radioisotope used for myocardial scan. The administered  activity of the isotope is 110MBe. How much is this activity in mCi?
\begin{enumerate}[label=(\alph*)]
\begin{multicols*}{2}
\item 1mCi
\item 2mCi
\item 3mCi
\item 4mCi
\item 5mCi
\end{multicols*}\flushright  {\small Ans: (c)}
\end{enumerate}

\item One symptom of mild radiation sickness is	
\begin{enumerate}[label=(\alph*)]
\item a lowered white cell count.
\item a lowered red blood cell count.
\item a raised white cell count.
\item a raised red blood cell count.
\item a white cell count of zero.
\flushright  {\small Ans: (a)}
\end{enumerate}

\item Alpha radiation is the most damaging because alpha particles	
\begin{enumerate}[label=(\alph*)]
\item have the largest charge.
\item have the greatest energy.
\item have the greatest mass.
\item consist of high energy electrons.
\item are damaging.
\flushright  {\small Ans: (c)}
\end{enumerate}

\item Gamma radiation is the most penetrating because gamma particles	
\begin{enumerate}[label=(\alph*)]
\item have the largest charge.
\item have the greatest energy.
\item have the greatest mass.
\item consist of high energy electrons.
\item are damaging.
\flushright  {\small Ans: (b)}
\end{enumerate}

\restoregeometry
\end{enumerate}
\end{multicols*}
\pagecolor{green!10}\afterpage{\nopagecolor}\newpage
\end{document}
