\documentclass[main.tex]{subfiles}
\begin{document}\newpage
\setdoublesep{0.35700 em}  % 'Bond Spacing'
\setatomsep{1.78500 em}    % 'Fixed Length'
\setbondoffset{0.18265 em} % 'Margin Width'
\newcommand{\bondwidth}{0.06642 em} % 'Line Width'
\setbondstyle{line width = \bondwidth}

 





%%%%%%%%%%%%HEADING
\begin{multicols}{2}
\begin{tcolorbox}[enhanced jigsaw,breakable,size=title,
colback=mybrown!05,colframe=black,fonttitle=\bfseries,
title=STUDENT INFO,pad at break=1mm, break at=15cm/0pt ]
\vspace{0.2cm}
\noindent Name: \rule{5cm}{0.4pt}Date:\rule{1cm}{0.4pt}\\
Pre-lab Done: \tikzcheckmark[scale=2,black]{no mark}\quad
\end{tcolorbox}
\end{multicols}
\hfill
\vspace{0.2cm}
\begin{center}
{\large \bfseries 
Pre-lab Questions 
\par
\Huge
Reaction Rates \&  Chemical Equilibrium
\\[5pt] \par}
\vspace{0.2cm}
\end{center}
\par
\noindent
\uline{  \hfill \normalsize \hfill       }
%%%%%%%%%%%%HEADING

\begin{enumerate}
% PELAB 1
\item Define speed of reaction.
\vspace{3cm}

\item Define forward reaction and reverse reaction.
\vspace{3cm}

\item For the following reaction, write down the forward reaction and the reverse reaction:
\begin{center}\ce{CO2(g) + H2O(l) <=> HCO3^{-}_{(Aq)} + H+(Aq)}\end{center}
\vspace{3cm}

%\item Explain the meaning of acidosis and alkalosis.
%\vspace{3cm}
%
%\item Explain why if you are hyperventilating you need to breathe into a paper bag.
%\vspace{3cm}




\end{enumerate}


\clearpage\mbox{}\clearpage



%%%%%%%%%%%%HEADING
\begin{multicols}{2}
\begin{tcolorbox}[enhanced jigsaw,breakable,size=title,
colback=mybrown!05,colframe=black,fonttitle=\bfseries,
title=STUDENT INFO,pad at break=1mm, break at=15cm/0pt ]
\vspace{0.2cm}
\noindent Name: \rule{5cm}{0.4pt}Date:\rule{1cm}{0.4pt}\\
Pre-lab Done: \tikzcheckmark[scale=2,black]{no mark}\quad
\end{tcolorbox}
\end{multicols}
\hfill
\vspace{0.2cm}
\begin{center}
{\large \bfseries 
Experiment
\par
\Huge
Reaction Rates \&  Chemical Equilibrium
\\[5pt] \par}
\vspace{0.2cm}
\end{center}
\par
\noindent
\uline{  \hfill \normalsize \hfill       }
%%%%%%%%%%%%HEADING

\vspace{0.2cm}{\large \bfseries 1. Factors that affect the chemical rate. Effect of temperature}
The goal of next three mini-experiments is to identify the factors that impact the rate of a chemical reaction. Reactions proceed at a certain rate, some are fast others are slow. By playing with a few factor you can increase the speed of a reaction generating more products in less time, or even slow down a reaction, avoiding the formation of products. You will study three different reaction and address the impact of three factors (1) the concentration of reactants, (2) temperature and (3) adding a catalyst on the chemical rate. This mini-experiment addresses the impact of temperature on reaction rate of the decomposition of sodium hydrogen carbonate:
\begin{center}\ce{NaHCO3(aq) + H3O+(aq) -> CO2(g) + 2H2O(l)}\end{center}
\begin{steps}
    \newstep[] Place 10mL of 0.1 M HCl in each of two different test tubes.
    \newstep[] Place one of the test tubes in a cold bath with ice--this is a 400mL beaker half-filled with ice and water. Cool the test tube to a temperature of $10^\circ$C.
        \newstep[]  Place the second test tube in a hot bath--this is a 400mL beaker half-filled with hot water from the tab. Warm up the test tube to a temperature close to $40^\circ$C.

            \newstep[] Remove both test tubes and place them in a test tube rack. Immediately, add one scoop of \ce{NaHCO3} (sodium bicarbonate or sodium hydrogencarbonate) to each tube. You will observe the appearance of bubbles. Write down which test tube produces bubbles first.
\end{steps}

\begin{center}\begin{tabular}{ |p{4cm}|p{8cm}|  }
\hline
      \begin{center}Test tube \end{center} &   \begin{center}Observation \end{center}         \\
         {\small(Hot/Cold)} &     {\small(intense/weak bubble formation) }        \\
\hline
   \vspace{0cm}\vspace{.25cm} &               \\\hline
   \vspace{0cm}\vspace{.25cm} &             \\\hline
   

\end{tabular}\end{center}




\newpage
 

\vspace{0.2cm}{\large \bfseries 2. Factors that affect the chemical rate. Effect of reactant concentration}
This mini-experiment addresses the impact of temperature on reaction rate of the reaction of magnesium with hydrochloric acid:
\begin{center}\ce{Mg(s) + 2HCl(aq) -> MgCl(aq) + H2(g)}\end{center}
\begin{steps}
    \newstep[] Place a 1-in piece of Mg in each of three different test tubes. Label each test tube 1 to 3.
    \newstep[] Measure 10mL of 1M HCl in a graduated cylinder and add it to test tube 1. Immediately, start recording the time and stop when all Mg has disappeared.
        \newstep[]  Repeat the previous step but now with 2M HCl and then 3M HCl. Write down the three different times in the table below.
\end{steps}
\begin{center}\begin{tabular}{ |p{4cm}|p{8cm}|  }
\hline
     Molarity of HCl&   Total Time           \\
\hline
   \vspace{0cm}\vspace{.25cm} &               \\\hline
   \vspace{0cm}\vspace{.25cm} &             \\\hline
      \vspace{0cm}\vspace{.25cm} &             \\\hline

\end{tabular}\end{center}
\vspace{0.2cm}{\large \bfseries 3. Factors that affect the chemical rate. Presence of a catalyst.}
This mini-experiment addresses the impact of catalysts on the reaction rate of the decomposition of hydrogen peroxide:
\begin{center}\ce{2H2O2(aq)  -> O2(g) + 2H2O(l)}\end{center}
You will add different possible catalysts into the reaction mixture and study where more oxygen bubbles are being produced.
\begin{steps}
    \newstep[] Place 2mL of 3\% \ce{H2O2} into each of five different test tubes. Label the test tubes from 1 to 5. Test tube 1 will be the reference test tube.
    \newstep[] Add a small spatula tip of \ce{MnO2} to test tube 2 and record your observations in comparison to test tube 1.  If you see more bubbles than on test tube 1 that would mean the substance you used is a catalyst.
        \newstep[]  Repeat the previous step now using a set of possible catalysts in the table below. Record your observations. If you see more bubbles than on test tube 1 that would mean the substance you used is a catalyst.
        \end{steps}
\begin{center}\begin{tabular}{ |p{4cm}|p{8cm}|p{4cm}|  }
\hline
    Test tube &   Observation (bubbles/no bubbles)  & Catalyst (yes/no)?         \\
\hline
   \vspace{0cm}Reference\vspace{.25cm} &           &    \\\hline
   \vspace{0cm}\ce{MnO2}\vspace{.25cm} &           &    \\\hline
   \vspace{0cm}\ce{Zn}\vspace{.25cm} &           &    \\\hline
      \vspace{0cm}Fresh potato\vspace{.25cm} &           &    \\\hline
      \vspace{0cm}Boiled potato\vspace{.25cm} &           &    \\\hline


\end{tabular}\end{center}



 



\newpage
 

\vspace{0.2cm}{\large \bfseries 4. Le Chatelier principle}
Reaction proceed from reactants to products but when products are former, reactions can also proceed from products to reactants. This establishes an equilibrium. When a reaction reached equilibrium, the forward and reverse reactions proceed at the same speed, so what is formed is also being consumed. You can alter a reaction in equilibrium pushing chemistry to the right of to the left so that mostly reactants or mostly products are being formed. You can do this by adding or removing reactants or by increasing or decreasing temperature. Le chatelier principle rationalizes the behavior of chemical reactions in equilibrium predicting the shift of the equilibrium. When reactants are added the reaction shifts to the right, when products are added the reaction differently shifts to the left. When reactants are removed, the reaction shifts to the left, and when products are removed it shifts to the right. In this mini-experiment you will address the impact of an equilibrium shift for the following reaction:

\begin{center}\ce{$\underbrace{\textcolor{black}{\ce{Fe^{3+}(aq) + SCN-(aq)}}}_{\textrm{yellow}}$ <=> $\underbrace{\textcolor{black}{\ce{FeSCN^{2+}(aq)}}}_{\textrm{red}}$}
\end{center}

\begin{steps}
    \newstep[] Measure 10mL of 0.01M \ce{Fe(NO3)3} and 10mL of 0.01M KSCN in a graduated cylinder. Pour both into a small beaker. Set up four test tubes in a rack add 3mL of previous mixture into each test tube. Label the test tubes from 1 to 4.
        \newstep[]  Test tube 1 will be the reference. Add 10 drops of water to this test tube. 
        \newstep[]  Add 10 drops of 1M \ce{Fe(NO3)3}--this is a product--to test tube 2. Record the color in comparison to test tube 1.
                \newstep[]  Add 10 drops of 1M \ce{KSCN}--this is a product--to test tube 3. Record the color in comparison to test tube 1.
        \newstep[]  Add 10 drops of 1M \ce{HCl} to test tube 4. This will remove Fe  by forming \ce{FeCl4-}. Record the color in comparison to test tube 1.

\end{steps}

\begin{center}\resizebox{18cm}{!} {\begin{tabular}{ |p{4cm}|p{4cm}|p{4cm}|p{4cm}|  }
\hline
      \begin{center}Test Tube\end{center} &  \begin{center}Color\end{center}  &  \begin{center}Color vs reference\end{center}  & \begin{center}Equilibrium shift\end{center}        \\
            {\small (1,2,3,4)} &   & {\small (Deeper or lighter)  }& {\small (\ce{->} or \ce{<-} )  }     \\

\hline
   \vspace{0cm} \vspace{.25cm} &     & N/A  &          \\\hline
   \vspace{0cm} \vspace{.25cm} &     &   &          \\\hline
   \vspace{0cm} \vspace{.25cm} &     &   &          \\\hline
   \vspace{0cm} \vspace{.25cm} &     &   &          \\\hline
   \vspace{0cm} \vspace{.25cm} &     &   &          \\\hline


\end{tabular}}\end{center}




\clearpage\mbox{}\clearpage

\newpage
%%%%%%%%%%%%HEADING
\begin{multicols}{2}
\begin{tcolorbox}[enhanced jigsaw,breakable,size=title,
colback=mybrown!05,colframe=black,fonttitle=\bfseries,
title=STUDENT INFO,pad at break=1mm, break at=15cm/0pt ]
\vspace{0.2cm}
\noindent Name: \rule{5cm}{0.4pt}Date:\rule{1cm}{0.4pt}\\
Pre-lab Done: \tikzcheckmark[scale=2,black]{no mark}\quad
\end{tcolorbox}
\end{multicols}
\hfill
\vspace{0.2cm}
\begin{center}
{\large \bfseries 
Post-lab Questions 
\par
\Huge
Reaction Rates \&  Chemical Equilibrium   
\\[5pt] \par}
\vspace{0.2cm}
\end{center}
\par
\noindent
\uline{  \hfill \normalsize \hfill       }
%%%%%%%%%%%%HEADING



 
\begin{enumerate}
\item{} What is the impact of temperature on the reaction rate? 
\vspace{1.5cm}
\item{} What is the impact of adding a catalyst on the reaction rate?
\vspace{1.5cm}
\item{} For the reaction below write down the expression of K$_c$ 
\begin{center}\ce{ H2CO3_{(aq)} <=> HCO3^{-}_{(Aq)} + H^+_{(Aq)}}\end{center}
\item{}  The chemical equilibrium that controls the PH of blood is
\begin{center}\ce{CO2_{(g)} + H2O_{(l)} <=> HCO3^{-}_{(Aq)} + H^+_{(Aq)}}\end{center}
Respiratory  alkalosis is caused by a lack of carbon dioxide in the blood that results from poor lung function or depressed breathing. When a patient has respiratory alkalosis, breathing from a paper bag can help. Based on the equilibrium, explain why this simple technique works.

\item{} In the miniexperiment 1, how did temperature affect the bubble production and why?\vspace{1.5cm}
\item{} In the miniexperiment 2, how did molarity affect the time for Mg to dissapear and why?\vspace{1.5cm}
\item{} In the miniexperiment 3, how do you explain the catalytic activity difference of fresh and boiled potatoes?\vspace{1.5cm}


\end{enumerate}


 


 


 





\end{document}
