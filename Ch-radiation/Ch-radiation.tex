\documentclass[main.tex]{subfiles}
\newcommand\chapterlabel{Ch-radiation}\setcounter{figurenewcounter}{0}\setcounter{tablenewcounter}{0}\setcounter{formulanewcounter}{0}\chapterpicture{../{\chapterlabel}/figure1}\chapterpicturelabel{PngImg}




\begin{document}
 

  
  
 \setcounter{chapter}{8}

\import{../\chapterlabel/files/}{ChapterName}



   
%         \begin{marginfigure}
%      \begin{tikzpicture} \node (a) at (0,0) {\includegraphics[width=4cm]{../{\chapterlabel}/figure1}} node[rotate=90, font=\tiny] at ([yshift=.5cm,xshift=.1cm]a.south east) {\textsuperscript{\textcopyright} PngImg} ;
%\end{tikzpicture}
%\end{marginfigure}


\import{../\chapterlabel/files/}{ChapterIntro}
 
%\begin{marginfigure}%LEARNING GOALS BOX
%\begin{mytcbox}{GOALS}
%\begin{enumerate}[label=\protect\circled{\color{white}\arabic*}]
%\import{../\chapterlabel/files/}{SectionGoal-The-nature-of-light}
%\import{../\chapterlabel/files/}{SectionGoal-The-atomic-line-spectra-Energy-levels-of-hydrogen}
%\import{../\chapterlabel/files/}{SectionGoal-The-atomic-line-spectra-Transition-energies}
%\import{../\chapterlabel/files/}{SectionGoal-Electronic-configuration-Full-electron-Configuration}
%\import{../\chapterlabel/files/}{SectionGoal-Periodic-Trends}
%
%\end{enumerate}
%\end{mytcbox}
%\vspace{1cm}
%\begin{tcolorbox}[enhanced,colback=red!5!white,colframe=black!50!red,boxrule=1pt,
%  arc=0pt,outer arc=0pt,drop heavy lifted shadow]
%\faGears\ 
%\import{../\chapterlabel/files/}{ChapterDiscussion}
%\end{tcolorbox} 
% \end{marginfigure}%LEARNING GOALS BOX

 
\import{../\chapterlabel/files/}{Quote-Schrodinger}








 
 
\section{The nature of light}\import{../\chapterlabel/files/}{SectionIntro-The-nature-of-light}
\sloppy\begin{description}
\item[\docfilehook{Light as a wave}{}]\import{../\chapterlabel/files/}{SubSection-The-nature-of-light-Light-as-a-wave}
\import{../\chapterlabel/files/}{Figure-Properties-of-waves}
\item[\docfilehook{The electromagnetic spectrum of light}{}]\import{../\chapterlabel/files/}{SubSection-The-nature-of-light-The-electromagnetic-spectrum-of-light}
\vspace{-1cm}\import{../\chapterlabel/files/}{Figure-Electromagnetic-field}
\item[\docfilehook{The double-slit experiment: light diffraction}{}] \import{../\chapterlabel/files/}{SubSection-The-nature-of-light-The-double-slit-experiment-light-diffraction}
\vspace{-1cm}\import{../\chapterlabel/files/}{Figure-Double-slit}
\vspace{0cm}\hspace{-5cm}\import{../\chapterlabel/files/}{Figure-waves-and-life}
\item[\docfilehook{Types and color of radiation}{}] \import{../\chapterlabel/files/}{SubSection-The-nature-of-light-Types-and-color-of-radiation}
\hspace{0cm}\vspace{0cm}\import{../\chapterlabel/files/}{Figure-Electromagnetic-spectra}
\import{../\chapterlabel/problems/}{SampleProblem2}
\import{../\chapterlabel/files/}{Table-Types-and-Color-or-radiation}
\end{description}

%%%%%%%%%%%%NOT FOR 121

\section{Properties of light} 
\sloppy\begin{description}
\item[\docfilehook{Frequency and energy}{}] \import{../\chapterlabel/files/}{SubSection-The-nature-of-light-Frequency-and-energy}
\item[\docfilehook{The speed of light}{}] \import{../\chapterlabel/files/}{SubSection-The-nature-of-light-The-speed-of-light}
\item[\docfilehook{Wavelength and energy}{}] \import{../\chapterlabel/files/}{SubSection-The-nature-of-light-Wavelength-and-energy}
\import{../\chapterlabel/problems/}{SampleProblem1}
\end{description}
%%%%%%%%%%%%NOT FOR 121





%%%%%%%%%%%%NOT FOR 121
\section{The photoelectric effect} 
\import{../\chapterlabel/files/}{SubSection-The-nature-of-light-Electron-Volt-a-new-unit-of-energy}
 \sloppy\begin{description}
\import{../\chapterlabel/files/}{Table-Workfunctions}
\item[\docfilehook{The photoelectric effect}{ }]\import{../\chapterlabel/files/}{SubSection-The-nature-of-light-The-photoelectric-effect}
\import{../\chapterlabel/problems/}{SampleProblem3}
\import{../\chapterlabel/files/}{Figure-Photoelectric-effect}
\end{description}
%%%%%%%%%%%%%%%%%%%%%%%%

 \section{The atomic line spectra}
\import{../\chapterlabel/files/}{SectionIntro-The-atomic-line-spectra}
\sloppy\begin{description}
\item[\docfilehook{Spectrum of atoms}{ }]\import{../\chapterlabel/files/}{SubSection-The-atomic-line-spectra-Spectrum-of-atoms}
\import{../\chapterlabel/files/}{Figure-Atomic-spectrum}
\item[\docfilehook{Atomic line spectrum of hydrogen}{}]\import{../\chapterlabel/files/}{SubSection-The-atomic-line-spectra-The-atomic-line-spectraAtomic-line-spectrum-of-hydrogen}
\import{../\chapterlabel/files/}{Figure-Spectrum-of-hydrogen}
\end{description}

%%%%%%%%%%%%%%NOT FOR 121
\section{Bohr's model}\import{../\chapterlabel/files/}{SubSection-The-atomic-line-spectra-The-Bohr-model}
\sloppy\begin{description}
\item[\docfilehook{Energy levels of hydrogen}{Energy levels of hydrogen}]\import{../\chapterlabel/files/}{SubSection-The-atomic-line-spectra-Energy-levels-of-hydrogen}
\vspace{-0.5cm}\import{../\chapterlabel/files/}{Figure-Energy-levels-of-hydrogen}
\import{../\chapterlabel/files/}{Table-Hydrogen-series}
\item[\docfilehook{Transition energies}{}]\import{../\chapterlabel/files/}{SubSection-The-atomic-line-spectra-Transition-energies}
\import{../\chapterlabel/files/}{Figure-Energy-transitions}
\item[\docfilehook{Bohn's formula for energy transitions}{ }]\import{../\chapterlabel/files/}{SubSection-The-atomic-line-spectra-Bohns-formula-for-energy-transitions}
\import{../\chapterlabel/problems/}{SampleProblem4}
\end{description}
%%%%%%%%%%%%%%NOT FOR 121

%%%%%%%%%%%%%NOT FOR 121
\section{The wave properties of matter}\import{../\chapterlabel/files/}{SubSection-The-atomic-line-spectra-The-wave-properties-of-matter}
 \hspace{-9em}\import{../\chapterlabel/files/}{Figure-Debroglie}
\import{../\chapterlabel/problems/}{SampleProblem5}
\sloppy\begin{description}
\item[\docfilehook{Electron diffraction}{ }]\import{../\chapterlabel/files/}{SubSection-The-atomic-line-spectra-Electron-diffraction}
\import{../\chapterlabel/files/}{Figure-Diffraction-patterns}
\item[\docfilehook{The uncertainty principle}{}] \import{../\chapterlabel/files/}{SubSection-Quantum-mechanics-and-electronic-structure-The-uncertainty-principle}
\import{../\chapterlabel/problems/}{SampleProblem7}
\end{description}
%%%%%%%%%%%% 


%%%%%%%%%%%NOT FOR 121
\section{Quantum mechanics}\import{../\chapterlabel/files/}{SectionIntro-Quantum-mechanics-and-electronic-structure}
\sloppy\begin{description}
\item[\docfilehook{Quantized energy and continuum energy}{}] \import{../\chapterlabel/files/}{SubSection-Quantum-mechanics-and-electronic-structure-Quantized-energy-and-continuum-energy}
\item[\docfilehook{The Schr\"{o}dinger equation}{}] \import{../\chapterlabel/files/}{SubSection-Quantum-mechanics-and-electronic-structure-The-Schrordinger}
\item[\docfilehook{The wave function: orbitals}{}] \import{../\chapterlabel/files/}{SubSection-Quantum-mechanics-and-electronic-structure-The-wave-function:-orbitals}
\end{description}
%%%%%%%%%%%NOT FOR 121


%%%%%%%%%%%NOT FOR 121
\section{Quantum numbers}\import{../\chapterlabel/files/}{SubSection-Quantum-mechanics-and-electronic-structure-Orbitals-are-described-by-three-quantum-numbers}
 \import{../\chapterlabel/files/}{Figure-Radial-distribution}  
\sloppy\begin{description}
\item[\docfilehook{Principal quantum number, $n$}{}]\import{../\chapterlabel/files/}{SubSection-Quantum-mechanics-and-electronic-structure-Principal-quantum-number-n}
\item[\docfilehook{Angular quantum number, $\ell$}{}]\import{../\chapterlabel/files/}{SubSection-Quantum-mechanics-and-electronic-structure-Angular-quantum-number-l}
\item[\docfilehook{Magnetic quantum number, $m_{\ell}$}{}]\import{../\chapterlabel/files/}{SubSection-Quantum-mechanics-and-electronic-structure-Magnetic-quantum-number-ml}
\item[\docfilehook{A fourth quantum number: the spin $m_s$}{}] \import{../\chapterlabel/files/}{SubSection-Quantum-mechanics-and-electronic-structure-A-fourth-quantum-number:-the-spin-ms}
\item[\docfilehook{The Pauli exclusion principle}{}] \import{../\chapterlabel/files/}{SubSection-Electronic-configuration-The-Pauli-exclusion-principle}
\item[\docfilehook{Radial distribution functions}{}]\import{../\chapterlabel/files/}{SubSection-Quantum-mechanics-and-electronic-structure-Different-orbitals-plots}
\import{../\chapterlabel/files/}{Table-quantum}
\import{../\chapterlabel/problems/}{SampleProblem6}
\item[\docfilehook{Shells and subshells (or levels and sublevels)}{}] \import{../\chapterlabel/files/}{SubSection-Quantum-mechanics-and-electronic-structure-Shells-and-subshells-or-levels-and-sublevels}
\item[\docfilehook{Orbital labels: $s$, $p$, $d$ and $f$}{ }] \import{../\chapterlabel/files/}{SubSection-Quantum-mechanics-and-electronic-structure-Orbital-labels}
 \import{../\chapterlabel/files/}{Figure-Shell-and-subshell}
\end{description}
%%%%%%%%%%% 

 

\section{Atomic orbitals}\import{../\chapterlabel/files/}{SectionIntro-Atomic-orbitals} 
\import{../\chapterlabel/files/}{Figure-Energy-levels-and-orbitals}

\sloppy\begin{description}
\item[\docfilehook{Energy levels}{}] \import{../\chapterlabel/files/}{SubSection-Quantum-mechanics-and-electronic-structure-Energylevels}
\import{../\chapterlabel/problems/}{SampleProblem8}
\item[\docfilehook{Energy sublevels}{}] \import{../\chapterlabel/files/}{SubSection-Quantum-mechanics-and-electronic-structure-Energy-sublevels}
\item[\docfilehook{Orbitals}{}] \import{../\chapterlabel/files/}{SubSection-Quantum-mechanics-and-electronic-structure-orbitals}
\item[\docfilehook{$s$ orbitals}{}] \import{../\chapterlabel/files/}{SubSection-Quantum-mechanics-and-electronic-structure-s-orbitals} 
 \import{../\chapterlabel/files/}{Figure-Orbitals-s}
\item[\docfilehook{$p$ orbitals}{}] \import{../\chapterlabel/files/}{SubSection-Quantum-mechanics-and-electronic-structure-p-orbitals}
 \import{../\chapterlabel/files/}{Figure-Orbitals-p}
\item[\docfilehook{$d$ and $f$ orbitals}{}] \import{../\chapterlabel/files/}{SubSection-Quantum-mechanics-and-electronic-structure-d-and-f-orbitals}
\import{../\chapterlabel/files/}{Figure-Orbitals-d}
\end{description}



\section{Electronic configuration of an atom}\import{../\chapterlabel/files/}{SectionIntro-Electronic-configuration-of-an-atom}
\sloppy\begin{description}
\stepcounter{figurenewcounter}   \refstepcounter{figure}  \label{Fig:{\chapterlabel}\thefigurenewcounter} 
\item[\docfilehook{Orbital Filling: the aufbau principle}{}]\import{../\chapterlabel/files/}{SubSection-Electronic-configuration-Orbital-Filling-the-aufbau-principle}
\item[\docfilehook{Full electron Configuration}{}] \import{../\chapterlabel/files/}{SubSection-Electronic-configuration-Full-electron-Configuration}
\import{../\chapterlabel/files/}{SideFigure-Orbital-filling-diagram}
\import{../\chapterlabel/problems/}{SampleProblem9}
\item[\docfilehook{Abbreviated Electron Configuration and orbital diagrams}{}] \import{../\chapterlabel/files/}{SubSection-Electronic-configuration-Abbreviated-Electron-Configuration-and-orbital-diagrams}
\import{../\chapterlabel/problems/}{SampleProblem10}
\item[\docfilehook{Hund's rule}{}] \import{../\chapterlabel/files/}{SubSection-Electronic-configuration-Hunds-rule}
\import{../\chapterlabel/problems/}{SampleProblem12}

\end{description}



\section{Monoatomic ions: cations and anions}\import{../\chapterlabel/files/}{SectionIntro-Monoatomic-ions}
\sloppy\begin{description}
\item[\docfilehook{Cations}{}]\import{../\chapterlabel/files/}{Subsection-Monoatomic-ions-cations}
\import{../\chapterlabel/files/}{Figure-Cations}
\item[\docfilehook{Anions}{}]\import{../\chapterlabel/files/}{Subsection-Monoatomic-ions-anions}
\import{../\chapterlabel/files/}{Figure-anions}
\item[\docfilehook{Ionic charges from group numbers}{}]\import{../\chapterlabel/files/}{Subsection-Atomics-Charges-from-group}

\import{../\chapterlabel/files/}{Table-Orbital-Ionic-charges-from-the-table}

\end{description}
\newpage



\section{Periodic Trends}\import{../\chapterlabel/files/}{SectionIntro-Periodic-Trends}
\vspace{0cm}\import{../\chapterlabel/files/}{Figure-Atomic-radius}

\sloppy\begin{description}
\item[\docfilehook{Effective charge}{}] \import{../\chapterlabel/files/}{SubSection-Periodic-Trends-Effective-charge}
\import{../\chapterlabel/files/}{Figure-Atomic-radius-plot}
\item[\docfilehook{Atomic radii}{}] \import{../\chapterlabel/files/}{SubSection-Periodic-Trends-Atomic-radius}
\import{../\chapterlabel/files/}{Table-Ionization-energies}

\import{../\chapterlabel/files/}{Figure-Ionization-energy-plot}
\item[\docfilehook{Ionization Energy}{}] \import{../\chapterlabel/files/}{SubSection-Periodic-Trends-Ionization-Energy}
\import{../\chapterlabel/files/}{Figure-Periodic-propeties}
\item[\docfilehook{Electron affinity}{}] \import{../\chapterlabel/files/}{SubSection-Periodic-Trends-Electron-affinity}
\item[\docfilehook{Electronegativity, EN}{}] \import{../\chapterlabel/files/}{SubSection-Periodic-Trends-ElectronegativityEN}
\item[\docfilehook{Metallic character }{Metallic character }]\import{../\chapterlabel/files/}{SubSection-Periodic-Trends-Metallic-character}
\import{../\chapterlabel/problems/}{SampleProblem11}
\item[\docfilehook{Ionic radius}{}]  \import{../\chapterlabel/files/}{SubSection-Periodic-Trends-Ionic-radius}
\end{description}


%%%%NOT FOR 121
\section{Photoelectron spectroscopy of atoms}
\import{../\chapterlabel/files/}{Figure-Photoelectron}%%%%NOT FOR 121
\import{../\chapterlabel/files/}{SubSection-Periodic-Trends-Photoelectron-spectroscopy-of-atoms}%%%%NOT FOR 121
\import{../\chapterlabel/files/}{Figure-PES-spectra}%%%%NOT FOR 121
\newpage
%%%%NOT FOR 121









 \checkoddpage\ifoddpage \clearpage\thispagestyle{empty}\mbox{}\clearpage \else  \fi \end{document}

