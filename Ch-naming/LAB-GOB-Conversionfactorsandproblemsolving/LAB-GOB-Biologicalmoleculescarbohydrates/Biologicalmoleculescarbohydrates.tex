\documentclass[main.tex]{subfiles}
\begin{document}\newpage
\setdoublesep{0.35700 em}  % 'Bond Spacing'
\setatomsep{1.78500 em}    % 'Fixed Length'
\setbondoffset{0.18265 em} % 'Margin Width'
\newcommand{\bondwidth}{0.06642 em} % 'Line Width'
\setbondstyle{line width = \bondwidth}
\definesubmol{x}{C(-[4]H)(-[0]OH)}
\definesubmol{y}{C(-[0]H)(-[4]HO)}
\definesubmol{k}{C(=[0]\textcolor{orange}{O})}
\definesubmol{a}{C(=[1]\textcolor{orange}{O})(-[3]H)}
\definesubmol{t}{C(-[2]H)(-[0]OH)(-[4]H)}







%%%%%%%%%%%%HEADING
\begin{multicols}{2}
\begin{tcolorbox}[enhanced jigsaw,breakable,size=title,
colback=mybrown!05,colframe=black,fonttitle=\bfseries,
title=STUDENT INFO,pad at break=1mm, break at=15cm/0pt ]
\vspace{0.2cm}
\noindent Name: \rule{5cm}{0.4pt}Date:\rule{1cm}{0.4pt}\\
Pre-lab Done: \tikzcheckmark[scale=2,black]{no mark}\quad
\end{tcolorbox}
\end{multicols}
\hfill
\vspace{0.2cm}
\begin{center}
{\large \bfseries 
Pre-lab Questions 
\par
\Huge
Biological Molecules: Carbohydrates
\\[5pt] \par}
\vspace{0.2cm}
\end{center}
\par
\noindent
\uline{  \hfill \normalsize \hfill       }
%%%%%%%%%%%%HEADING

\begin{enumerate}
% PELAB 1

\item Identify the functional groups in the following biomolecules:
\begin{center}\resizebox{18cm}{!} {\begin{tabular}{ |p{4cm}|p{4cm}|p{4cm}| m{4cm}| }
\hline
Molecule &  Functional Group   &Molecule  &Functional Group       \\
\hline
\vspace{0cm}\begin{center} \glucose[color={anomerO}{orange}]\end{center}\vspace{0.8cm} &  &\vspace{0.8cm}\hspace{0.2cm}\setatomsep{2.5em}\glucose[model=haworth,ring, color={anomerH}{orange}, color={anomerO}{orange}, color={ringO}{red}]     &  \\
\hline
\end{tabular}}\end{center}

\item Calculate the molecular formula for the following biomolecules:
\begin{center}\resizebox{18cm}{!} {\begin{tabular}{ |p{4cm}|p{4cm}|p{4cm}| m{4cm}| }
\hline
Molecule &  Expanded Formula   &Molecule  & Expanded Formula       \\
\hline
\vspace{0.4cm}\hspace{0.2cm}\setatomsep{2.5em}\glucose[model=haworth,ring, color={anomerH}{orange}, color={anomerO}{orange}, color={ringO}{red}]  &  &   \begin{center}\setatomsep{2.5em}\chemfig{HO-[:-90,0.5,2]?-[:60,1.0](-[:90,0.7]\textcolor{orange}{OH})(-[:-90,0.7]-[:-30,0.5]OH)-[:160,1.2]\textcolor{red}{O}-[:-160,1.2](-[:90,0.7](-[:130,0.5]OH))-[:-60, 1.0]?(-[:-90,0.5]OH)}
\end{center}  &  \\
\hline
\end{tabular}}\end{center}

\item Calculate the expanded formula for the following biomolecules. Remember each carbon atom has to be connected to a total of four bonds:
\begin{center}\resizebox{18cm}{!} {\begin{tabular}{ |p{4cm}|p{4cm}|p{4cm}| m{4cm}| }
\hline
Molecule &  Expanded Formula   &Molecule  & Expanded Formula       \\
\hline
\vspace{0.2cm}\hspace{0.2cm}\glucose[model={fischer=skeleton },color={anomerO}{orange}, color={O-C5}{red}, color={H-C5}{red}] \vspace{0.1cm}&  &   \vspace{0.2cm}\hspace{0.2cm}\chemfig{[2]OH-[4](-(-[:0, 1.2]\textcolor{red}{OH})-(-[:0]HO)-(-[:180]OH)-(=[0]\textcolor{orange}{O})-(-[0]OH))}
\vspace{0.1cm}
  &  \\
\hline
\end{tabular}}\end{center}



\end{enumerate}


\clearpage\mbox{}\clearpage



%%%%%%%%%%%%HEADING
\begin{multicols}{2}
\begin{tcolorbox}[enhanced jigsaw,breakable,size=title,
colback=mybrown!05,colframe=black,fonttitle=\bfseries,
title=STUDENT INFO,pad at break=1mm, break at=15cm/0pt ]
\vspace{0.2cm}
\noindent Name: \rule{5cm}{0.4pt}Date:\rule{1cm}{0.4pt}\\
Pre-lab Done: \tikzcheckmark[scale=2,black]{no mark}\quad
\end{tcolorbox}
\end{multicols}
\hfill
\vspace{0.2cm}
\begin{center}
{\large \bfseries 
Experiment
\par
\Huge
Biological Molecules: Carbohydrates
\\[5pt] \par}
\vspace{0.2cm}
\end{center}
\par
\noindent
\uline{  \hfill \normalsize \hfill       }
%%%%%%%%%%%%HEADING

\vspace{0.2cm}{\large \bfseries Stereoisomers: d and l }
Use the molecular models set for this experiment. Each sphere represents an element. Carbon is black, hydrogen white, oxygen red and nitrogen blue. Short bonds are simple bonds, whereas long bonds are used to construct double bonds--you need two of the long bonds to make a double bond. Build up the following molecules and compare your molecule with the other's molecule in class. Are all molecules the same? Show your professor all molecular models before proceeding to next part.
%\begin{center}\chemfig{[4]C(-!x-!a)}\end{center}
\begin{center}\chemfig{[2]Cl-!x-!a}\end{center}

\vspace{0.2cm}{\large \bfseries Aldo and keto monosaccharides}
Use the molecular models set for this experiment. Each sphere represents an element. Carbon is black, hydrogen white, oxygen red and nitrogen blue. Short bonds are simple bonds, whereas long bonds are used to construct double bonds--you need two of the long bonds to make a double bond. Build up the following molecules. Show your professor all molecular models before proceeding to next part. Indicate whether the monosaccharides are aldose of ketose depending on the nature of the functional groups in the molecule. Indicate if the molecules is \emph{d} or  \emph{l} depending on the location of the second OH group starting from the bottom of the molecule: right is \emph{d} and left is \emph{l}.
\begin{center}\resizebox{18cm}{!} {\begin{tabular}{ |p{4cm}|p{4cm}|p{4cm}| m{4cm}| }
\hline
Molecule &  Skeletal Formula   &Molecule  & Skeletal  Formula       \\
\hline
\vspace{0.1cm}\hspace{0.4cm}\glucose[color={anomerO}{orange}, color={O-C5}{red}, color={H-C5}{red}]\vspace{0.1cm}&  &   \vspace{0.1cm}\hspace{0.4cm}\chemfig{[2]OH-[4]C(-[6]H)(-[4]H)(-C(-[4]\textcolor{red}{OH})(-[0, 1.2]H)-!x-!y-!k-!t)}
\vspace{0.2cm}
  &  \\
\hline
Aldo or Keto?\vspace{0.4cm} &  &Aldo or Keto?\vspace{0.4cm}  &  \\
\hline
\emph{d} or \emph{l}?\vspace{0.4cm} &  &\emph{d} or \emph{l}?\vspace{0.4cm}  &  \\
\hline
\end{tabular}}\end{center}



 
\newpage
 \vspace{0.2cm}{\large \bfseries Haworth cyclic structures of monosaccharides}
Monosaccharides presents a linear structure in dried form. In water they cycle forming Haworth cyclic structures. One can obtain the cyclic form from the linear form.\\
Use the molecular models set for this experiment. Each sphere represents an element. Carbon is black, hydrogen white, oxygen red and nitrogen blue. Build up the following molecules and complete the table. The Haworth structure will be $\alpha$ or $\beta$ depending on the position of the OH group of the carbon directly connected to the \ce{-O-} bond: $\alpha$ when it points down and $\beta$ when it points up.
Show your professor all molecular models before proceeding to next part.
\begin{center}\resizebox{18cm}{!} {\begin{tabular}{ |p{4cm}|p{4cm}|p{4cm}| m{4cm}| }
\hline
Molecule &  Expanded Formula   &Molecule  & Expanded  Formula       \\
\hline
\vspace{0.1cm}\hspace{0.4cm}\carbohydrate[model={fischer=skeleton}, color={anomerO}{orange},color={H-C5}{red}, color={O-C5}{red} ]{llrr}\vspace{0.1cm}&  &   \vspace{0.1cm}\hspace{0.4cm}\setatomsep{2.5em}\glucose[model=haworth,ring,  color={ringO}{red},color={anomerO}{orange}, color={anomerH}{orange}]
\vspace{0.2cm}
  &  \\
\hline
\emph{d} or \emph{l}?\vspace{0.4cm} &  &$\alpha$ or $\beta$?\vspace{0.4cm}  &  \\
\hline
\end{tabular}}\end{center}
In order to obtain a cycle you just need to get the linear structure and turn in clockwise 90 degree until is horizontal, the carbonyl or keto will fall in the right part of the structure and the last C atom in the left part. After that you need to draw the structure in a boat shape, as the OH from carbon number 5 will attach carbon number one and produce a cycle.
\begin{center}
\begin{tabular*}{1\textwidth}{@{\extracolsep{\fill}}>{\centering}m{0.2\textwidth}>{\centering}m{0.4\textwidth}>{\centering}m{0.3\textwidth}>{\centering}m{0.4\textwidth}}
\carbohydrate[model={fischer=skeleton}, color={anomerO}{orange},color={H-C5}{red}, color={O-C5}{red} ]{llrr}&
\setatomsep{2.5em}\chemfig{\chemabove[1ex]{}{\tiny \textcolor{red}{6}}
(-[-2]OH)(-\chemabove[1ex]{}{\tiny \textcolor{red}{5}}
(-[-2]\textcolor{red}{O}\textcolor{red}{H})-\chemabove[1ex]{}{\tiny \textcolor{red}{4}}
(-[-2]OH)-\chembelow[1ex]{}{\tiny \textcolor{red}{3}}
(-[2]OH)-\chembelow[1ex]{}{\tiny \textcolor{red}{2}}
(-[2]OH)-\chemabove[1ex]{}{\tiny \textcolor{red}{1}}
(=[-1]\textcolor{orange}{O}))}&
\setatomsep{2.5em}\glucose[model=chair,color={anomerO}{orange},color={O-C5}{red},color={H-C5}{red}]&
\setatomsep{2.5em}\glucose[model=haworth,ring,  color={ringO}{red},color={anomerO}{orange}, color={anomerH}{orange}]
\tabularnewline\addlinespace
\textcolor{blue}{D-Mannose} & \textcolor{blue}{D-Mannose rotated} & \textcolor{blue}{D-Mannose as a boat}& \textcolor{blue}{D-Mannose } \tabularnewline
\end{tabular*}\end{center}


Now, try yourself and create the cycled Haworth form of:

\begin{center}\resizebox{18cm}{!} {\begin{tabular}{ |p{4cm}|p{4cm}|p{4cm}| m{4cm}| }
\hline
Molecule &  Skeletal Formula   &Haworth & Expanded Haworth      \\
\hline
\vspace{0.1cm}\hspace{0.4cm}\carbohydrate[hexose, color={anomerO}{orange}, color={O-C5}{red}, color={H-C5}{red}]{rlrr}&  &   \vspace{0.1cm}\vspace{0.1cm}\hspace{0.4cm}\setatomsep{2.5em}
\vspace{0.2cm}
  &  \\
\hline
\emph{d} or \emph{l}?\vspace{0.4cm} &  &$\alpha$ or $\beta$?\vspace{0.4cm}  &  \\
\hline
\end{tabular}}\end{center}



 

\newpage
 \vspace{0.2cm}{\large \bfseries Haworth cyclic structures of monosaccharides using the molecular models}
Use the molecular models set for this experiment. Each sphere represents an element. Carbon is black, hydrogen white, oxygen red and nitrogen blue. 
Build up only one of the following linear molecules and complete the table. The professor will tell you which one to work on. After that cycle the linear molecule by connecting carbon number one (the ketone carbon) and carbon number five by means of a \ce{-O-} bridge. A molecule of water will be produce. Show your professor all molecular models before proceeding to next part. 
\begin{center}\resizebox{18cm}{!} {\begin{tabular}{ |p{4cm}|p{4cm}|p{4cm}| m{4cm}| }
\hline
Expanded Formula &  Skeletal Formula   &Haworth & Expanded Haworth      \\
\hline
&\vspace{0.1cm}\hspace{0.4cm}\carbohydrate[model={fischer=skeleton}, color={anomerO}{orange},color={H-C5}{red}, color={O-C5}{red} ]{rlrr}  &   \vspace{0.1cm}\vspace{0.1cm}\hspace{0.4cm}
  &  \\
\hline
\emph{d} or \emph{l}?\vspace{0.4cm} &  &$\alpha$ or $\beta$?\vspace{0.4cm}  &  \\
\hline
Expanded Formula &  Skeletal Formula   &Haworth & Expanded Haworth      \\
\hline
&\vspace{0.1cm}\hspace{0.4cm}\carbohydrate[model={fischer=skeleton}, color={anomerO}{orange},color={H-C5}{red}, color={O-C5}{red} ]{lrlr}  &   \vspace{0.1cm}\vspace{0.1cm}\hspace{0.4cm}
  &  \\
\hline
\emph{d} or \emph{l}?\vspace{0.4cm} &  &$\alpha$ or $\beta$?\vspace{0.4cm}  &  \\
\hline
Expanded Formula &  Skeletal Formula   &Haworth & Expanded Haworth      \\
\hline
&\vspace{0.1cm}\hspace{0.4cm}\carbohydrate[model={fischer=skeleton}, color={anomerO}{orange},color={H-C5}{red}, color={O-C5}{red} ]{llll}  &   \vspace{0.1cm}\vspace{0.1cm}\hspace{0.4cm}
  &  \\
\hline
\emph{d} or \emph{l}?\vspace{0.4cm} &  &$\alpha$ or $\beta$?\vspace{0.4cm}  &  \\
\hline
\end{tabular}}\end{center}


After that, you will connect your cyclic structure with another's team monosaccharides by means of a glycosidic bond. 


\begin{center}\resizebox{18cm}{!} {\begin{tabular}{ |p{4cm}|p{4cm}|p{8cm}| }
\hline
Haworth A &  Haworth B    &Disaccharaide   \\
\hline
&&\vspace{1cm}\vspace{1cm}\vspace{1cm}
    \\
\hline
$\alpha$ or $\beta$?\vspace{0.4cm} &  $\alpha$ or $\beta$?\vspace{0.4cm} &  Type of Glycosidic bond?\vspace{0.4cm}  \\
\hline
\end{tabular}}\end{center}






 


\newpage
 
\vspace{0.2cm}{\large \bfseries Disaccharide}
For the following disaccharides identify the type of glycosidic bond and the anomer type ($\alpha$ or $\beta$).
\begin{center}\resizebox{18cm}{!} {\begin{tabular}{ |p{7cm}|p{7cm}|p{2cm}|  }
\hline
    Disaccharide & Glycosidic bond & anomer    \\
\hline
 \begin{center}\chemname{\chemfig{HO-[:-90,0.8,2]?[b]-[:-30,1.3](-[:90,0.8]OH)-[0,1.3](-[:-90,0.8]OH)-[:30,1.3](-[:90,0.8]-[:-45,1.5]\textcolor{orange}{O}-[:-45,1.2]-[:90,0.8]?[a]-[:-30,1.3](-[:90,0.8]OH)-[0,1.3](-[:-90,0.8]OH)-[:30,1.3](-[:-90,0.8]\textcolor{orange}{OH})-[:150,1.2]\textcolor{red}{O}-[4, 1.3]?[a](-[:90, 1.0]-[:160, 0.7]OH))-[:150,1.2]\textcolor{red}{O}-[4, 1.3]?[b](-[:90, 1.0]-[:160, 0.7]OH)} }{\textcolor{blue}{lactose}}\end{center}&     &    \\
\hline
\begin{center}\chemname{\chemfig{HO-[:-90,0.8,2]?[b]-[:-30,1.3](-[:90,0.8]OH)-[0,1.3](-[:-90,0.8]OH)-[:30,1.3](-[:-90,0.8]-[:0,1.5]\textcolor{orange}{O}-[:0,1.2]-[:90,0.8]?[a]-[:-30,1.3](-[:90,0.8]OH)-[0,1.3](-[:-90,0.8]OH)-[:30,1.3](-[:-90,0.8]\textcolor{orange}{OH})-[:150,1.2]\textcolor{red}{O}-[4, 1.3]?[a](-[:90, 1.0]-[:160, 0.7]OH))-[:150,1.2]\textcolor{red}{O}-[4, 1.3]?[b](-[:90, 1.0]-[:160, 0.7]OH)}  }{\textcolor{blue}{maltose}}\end{center}&     &    \\
\hline
\begin{center}\chemfig[][scale=1]{HO-[:-90,0.8,2]?[b]-[:-30,1.3](-[:90,0.8]OH)-[0,1.3](-[:-90,0.8]OH)-[:30,1.3](-[:90,0.8]-[:0,1.5]\textcolor{orange}{O}-[:0,1.2]-[:-90,0.8]?[a]-[:-30,1.3](-[:90,0.8]OH)-[0,1.3](-[:-90,0.8]OH)-[:30,1.3](-[:-90,0.8]\textcolor{orange}{OH})-[:150,1.2]\textcolor{red}{O}-[4, 1.3]?[a](-[:90, 1.0]-[:160, 0.7]OH))-[:150,1.2]\textcolor{red}{O}-[4, 1.3]?[b](-[:90, 1.0]-[:160, 0.7]OH)} \end{center}
&     &    \\
\hline



\end{tabular}}\end{center}
 


\end{document}