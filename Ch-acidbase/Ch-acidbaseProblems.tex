\documentclass[main.tex]{subfiles}
\newcommand\chapterlabel{Ch-acidbase}
\begin{document}\newpage
 
\newgeometry{left=0.8in,right=0.8in, top=2.5cm,bottom=2cm}
\fancyhfoffset[E,O]{0pt}
\setlength{\columnsep}{30pt}
\begin{conclusion}
\end{conclusion}
%\setstretch{0.3}
\begin{multicols*}{2}\setcounter{numA}{1}
{\raggedright\textsc{\textbf{The nature of acids and bases}}\par}
 \import{../\chapterlabel/problems/problems/}{Problems-The-nature-of-acids-and-Bases-Arrhenius-acid-base-model}
 \import{../\chapterlabel/problems/problems/}{Problems-The-nature-of-acids-and-Bases-Bronsted-acids}
 % % %%%%%%%%%%%%%NOT FOR 121
 \import{../\chapterlabel/problems/problems/}{Problems-The-nature-of-acids-and-Bases-Lewis-acid-base-model}
 % %%%%%%%%%%%%%NOT FOR 121

{\raggedright\textsc{\textbf{Dissociations of acids \& bases}}\par}
 \import{../\chapterlabel/problems/problems/}{Problems-Strength-of-acids-and-bases-Strength-of-acids-and-bases-}
 \import{../\chapterlabel/problems/problems/}{Problems-Dissociation-of-acids-and-bases-Conjugate-acids-and-bases}
 
{\raggedright\textsc{\textbf{The PH scale}}\par}
 \import{../\chapterlabel/problems/problems/}{Problems-The-PH-scale-Protons-and-Hydroxyls}
 \import{../\chapterlabel/problems/problems/}{Problems-The-PH-scale-The-PH-scale}
 
 %%%%%%%%%%%%%NOT FOR 121
  {\raggedright\textsc{\textbf{PH of strong acid-base solutions}}\par}
 \import{../\chapterlabel/problems/problems/}{Problems-The-PH-scale-PH-of-strong-electrolyte-solutions} 
  %%%%%%%%%%%%%NOT FOR 121

 
 % %%%%%%%%%%%%%NOT FOR 121
 {\raggedright\textsc{\textbf{PH of weak acid-base solutions}}\par}
 \import{../\chapterlabel/problems/problems/}{Problems-The-PH-scale-PH-of-solutions-of-weak-acids-and-bases} 
 \import{../\chapterlabel/problems/problems/}{Problems-The-PH-scale-PH-of-solutions-of-weak-acids-and-bases-Solutions-of-acids-and-bases} 
% %%%%%%%%%%%%%NOT FOR 121
	


{\raggedright\textsc{\textbf{Buffer solutions}}\par}
 \import{../\chapterlabel/problems/problems/}{Problems-Buffer-solutions-Buffers}
  \import{../\chapterlabel/problems/problems/}{Problems-Buffer-solutions-PH-of-a-Buffer-solution}
  % % %%%%%%%%%%%%%NOT FOR 121
 \import{../\chapterlabel/problems/problems/}{Problems-Buffer-solutions-PH-of-Buffer-solution-mixed-with-acids-or-bases} 
 \import{../\chapterlabel/problems/problems/}{Problems-Buffer-solutions-PH-of-a-Buffer-solution-with-weak-acid} 
% % %%%%%%%%%%%%%NOT FOR 121

{\raggedright\textsc{\textbf{Titrations}}\par}
 \import{../\chapterlabel/problems/problems/}{Problems-Titrations-Endpoint-formula}
 
 %  % %%%%%%%%%%%%%NOT FOR 121
 {\raggedright\textsc{\textbf{Titration curves}}\par}
 \import{../\chapterlabel/problems/problems/}{Problems-Titrations-Titration-curves}
 {\raggedright\textsc{\textbf{Quantitative analysis of a titration}}\par}
 \import{../\chapterlabel/problems/problems/}{Problems-Titrations-Titration-PH-formulas} 
 {\raggedright\textsc{\textbf{Molecular mechanisms behind acid-base strength}}\par} 
 \import{../\chapterlabel/problems/problems/}{Problems-Strength-of-acids-and-its-structure} 
 {\raggedright\textsc{\textbf{Indicators}}\par} 
 \import{../\chapterlabel/problems/problems/}{Problems-Indicators} 
{\raggedright\textsc{\textbf{Titrations of polyprotic acids}}\par} 
 \import{../\chapterlabel/problems/problems/}{Problems-Titrations-of-polyprotic-acids} 
 % %%%%%%%%%%%%%NOT FOR 121

\end{multicols*}


%%%%%%%%%%%%%%%%%NEW ANSWER ENVIRONMENT
\newpage \begin{answerbox}
\begin{answersenvironment}
 \begin{localsize}{10}
{\Large \bf Answers}
\SetupExSheets{ headings = inline-nr , counter-format = qu) ,}
\printsolutions 
%  \printsolutions[byID={1,3,5,7,9,11,13,15,17,19,21,23,25,27,29,31,33,35,37, 39, 41, 43, 45, 47, 49, 51,53,55,57,59}]
%   \printsolutions[byID={1,3,5,7,9,11,13,15,17,19,21,23,25,27,29,31  }] %%% FOR 121
 \end{localsize}
 \end{answersenvironment}
\end{answerbox}
%%%%%%%%%%%%%%%%%NEW ANSWER ENVIRONMENT






\end{document}
